\documentclass{beamer}
\usepackage[utf8]{inputenc}
\usepackage{listings}
\usepackage{graphicx}
\usetheme{Madrid}
\usecolortheme{beaver}

\input{revision}

\title{OTP News}
\subtitle{Since last meetup}
\author[joaohf]{João Henrique Ferreira de Freitas \\ \texttt{https://github.com/joaohf} \\ \texttt{joaof@gmail.com}}
\date[EMC May 2018]{Elixir Meetup, Campinas May 2018}

\logo{\Revision}

%\AtBeginSection[]
%{
%  \begin{frame}
%    \frametitle{Table of Contents}
%    \tableofcontents[currentsection]
%  \end{frame}
%}

\begin{document}
  \begin{frame}
    \titlepage
  \end{frame}

  \section[Section]{Community}
  
  \begin{frame}
    \frametitle{Community}
    \begin{itemize}
      \item \href{https://www.youtube.com/channel/UC47eUBNO8KBH_V8AfowOWOw}{Youtube: Code BEAM}
      \item \href{https://www.youtube.com/channel/UCKrD_GYN3iDpG_uMmADPzJQ}{Youtube: Erlang Solutions}      
      \item Slack channels: Elixir and Erlang
      \item Elixir Forum
      \item Erlang Questions ML
    \end{itemize}
  \end{frame}
  
  \begin{frame}
    \frametitle{OpenErlang}
    \framesubtitle{Celebrating 20 years of open-sourced Erlang!}

    \begin{block}{}
    Since Erlang OTP was open sourced in 1998.
    \begin{figure}[t]
    \includegraphics[scale=0.1]{img/20.png}
    \centering
    \end{figure}
    \end{block}
    
    \pause
    
    \begin{alertblock}{}   
    \href{https://www.youtube.com/watch?v=j6wbuV8pMx8}{Miriam Pena - Keynote: Unsung Heroes of the BEAM - Code BEAM SF 2018}
    \end{alertblock}
  \end{frame}
  
  \begin{frame}
    \frametitle{OTP 21 Highlights}

    \begin{block}{}
      \begin{itemize}
        \item \href{http://blog.erlang.org/My-OTP-21-Highlights}{Erlang blog: My-OTP-21-Highlights}
        \item \href{https://github.com/erlang/otp/releases/tag/OTP-21.0-rc1}{Erlang 21 RC-1 relase notes}
        \item \href{https://www.youtube.com/watch?v=hHhm0bfdj-4}{Raimo Niskanen - Update OTP Team - Code BEAM SF 2018}   
      \end{itemize}
    \end{block}
     
    \pause
    
    \begin{itemize}[<+->]
    \item Compiler and Interpreter
      \begin{itemize}
      \item Code such as f({ok, Val}) -> {ok, Val} is now automatically rewritten to f({ok, Val} = Tuple) -> Tuple
      \item 20\% code size reduced
      \item 5\% performance increased
      \end{itemize}    

    \item ERTS
      \begin{itemize}
      \item file handling port to nifs
      \item I/O Polling: rewriter to use modern OS kernel polling features
      \end{itemize}

    \item Distribution: use your own carrier
    \item Logger: new logging framework (inpired by lager, Elixir Logger, Python logger)
    \item Stacktrace
    \end{itemize}
    
  \end{frame}
  
  \begin{frame}
    \frametitle{Elixir 1.6.x}
    
    \begin{block}{}
      \begin{itemize}
      \item \href{https://elixir-lang.org/blog/2018/01/17/elixir-v1-6-0-released}{Elixir v1.6.0 Released}
      \item \href{https://github.com/elixir-lang/elixir/releases/tag/v1.6.0}{Relase Notes - 1.6.0}
      \end{itemize}
    \end{block}

    \pause
    
    \begin{itemize}[<+->]
      \item Code formatter
      \begin{itemize}
        \item \href{https://www.youtube.com/watch?v=x2ckfhqB9nA}{Introducing HDD: Hughes Driven Development - José Valim - Elixir Conf EU 2018}
        \item "The Design of a Pretty-Printing Library"
      \end{itemize}
      \item DynamicSupervisor, encapsulates simple\_one\_for\_one strategy and APIs in a proper module
      \item New function attributes: @deprecated and @since
      \item defguard and defguardp    
    \end{itemize}

  \end{frame}  

  \begin{frame}
    \frametitle{My Highlights}
    \framesubtitle{Docker}
    
    \begin{block}{}
        Docker Erlang Oficial: \href{https://github.com/erlang/docker-erlang}{Docker Erlang (deprecated)} to \href{https://github.com/erlang/docker-erlang-otp}{Docker Erlang OTP}
    \end{block}
    
  \end{frame}
  
  \begin{frame}
    \frametitle{My Highlights}
    \framesubtitle{Riak Core}

    \begin{block}{What is?}   
    \begin{quote}
    Riak Core is the distributed systems framework that forms the basis of how Riak distributes data and scales. More generally, it can be thought of as a toolkit for building distributed, scalable, fault-tolerant applications.        
    \end{quote}
    \end{block}
    
    \pause
    
    \begin{block}{}
    \begin{itemize}
      \item \href{https://github.com/basho/riak_core}{Basho Riak Core}
      \item \href{http://marianoguerra.org/posts/riak-core-tutorial-part-1-setup.html}{Riak Core Tutorial Part 1: Setup}
      \item \href{https://www.youtube.com/watch?v=1qyjAU81Qhg}{Sometimes the parts are greater than their sum}
    \end{itemize}

    \end{block}
    
  \end{frame}

  \begin{frame}
    \frametitle{Past Conferences}
    
    \begin{itemize}
      \item \href{https://codesync.global/conferences/code-beam-sf-2018}{Code BEAM SF}
      \begin{itemize}
      \item \href{https://www.erlang-solutions.com/blog/elixirconf-eu-2018-highlights.html}{Check the highlights here}
      \end{itemize}
      \item \href{https://codesync.global/conferences/code-beam-lite-milan-2018}{Code BEAM Lite Milan}
      \item \href{https://www.erlang-solutions.com/blog/elixirconf-eu-2018-highlights.html}{ElixirConf EU 2018}
    \end{itemize}
  \end{frame}
  
  \begin{frame}
    \frametitle{Future Conferences}

    \begin{itemize}
      \item \href{https://codesync.global/conferences/code-elixir-2018}{Code Elixir 16 August 2018}
      \item \href{https://codesync.global/conferences/code-beam-sto-2018}{Code BEAM STO 31 May 2018}
      \item \href{https://codesync.global/conferences/code-mesh-2018}{Code Mesh 08 November}    
    \end{itemize}
  \end{frame}

  \begin{frame}
    \begin{center}
    \Huge Thank You!
    \end{center}
  \end{frame}
    
  \begin{frame}
    \frametitle{Colophon}
    
    \begin{itemize}
    \item Latex
    \item Beamer
    \item Texmaker
    \end{itemize}
    
  \end{frame}
  
  
  
% etc
\end{document}
